% @author: Felix Hekhorn
\documentclass[
  english,		% Sprache
  a4paper,		% Papierformat
  11pt,			% Schriftgröße (default 10pt)
  DIV=12,		% Seiteneinteilung
  titlepage,
  toc=bibnumbered,
  parskip=full,  	% Absätze (full,half,false -+*)
  headings=normal,
  BCOR=12mm,
  numbers=noenddot
]{scrartcl}
%\documentclass[a4paper,10pt]{article}
\usepackage{scrtime,scrlfile,scrpage2}

\usepackage[status=draft]{fixme}
%\usepackage[status=final]{fixme}

\usepackage[utf8]{inputenc}
%\usepackage[ngerman]{babel} % Sprache
\usepackage[ngerman,english,main=english]{babel} % Sprache
\selectlanguage{english}
\usepackage{amsmath, amssymb}

%\usepackage{graphicx} % Grafiken einbinden
% The following is needed in order to make the code compatible
% with both latex/dvips and pdflatex.
\ifx\pdftexversion\undefined{}
\usepackage[dvips]{graphicx}
\else
\usepackage[pdftex]{graphicx}
\DeclareGraphicsRule{*}{mps}{*}{}
\fi
\usepackage{pdfpages}

%\usepackage[left=2cm,right=2cm,top=2.5cm,bottom=3cm]{geometry}
\usepackage{color}
\usepackage{bbm}
\usepackage{csquotes}

\usepackage{pdflscape}
\usepackage{ulem}
\usepackage{url}
\usepackage{caption}
\usepackage{subcaption}
\usepackage{array}
\usepackage{multirow}
\usepackage{listings}
\usepackage{placeins}

\usepackage{siunitx} % SI Einheiten
\usepackage[version=3]{mhchem}
%\sisetup{
%	exponent-product = \!\cdot\!,
%	output-product = \cdot,
%	list-final-separator =  { und } ,
%	list-pair-separator = { und } ,
%	range-phrase = { bis },
%	output-decimal-marker = {,},
%	separate-uncertainty = true,
%	group-digits = false
%}

%\usepackage{simplewick}
%\usepackage{feynmf}
\usepackage{slashed}
\usepackage{hepnames}

%\usepackage{biblatex}
\usepackage[numbers]{natbib}
%\bibliographystyle{natdin}
%\bibliographystyle{kp}
\bibliographystyle{utphys}


\begin{document}
here $g = pdf$
\begin{align}
g(z) =  \sum\limits_k p_k(z) g(x_k)
\end{align}

\[ p_k(x_j) = \delta_{jk} \]
\[   \]
\begin{align}
|f\rangle  &= \sum_k c_k | g_k \rangle\\
c_k &= f(x_k)
\end{align}

\begin{align}
F(x) &= \int\limits_{x}^1 \frac{dz}{z} c(z) g(z)\\
F(x_j) &= \sum\limits_k \int\limits_{x_j}^1 \frac{dz}{z} c(z) p_k(z) g(x_k)
\end{align}

here $g' = arbitrary fnc$
\begin{align}
h_2(\xi) &= \int\limits_{\xi}^1 \frac{du}{u} g'(u) F(u)\\
h_2(\xi) &= \sum_j \int\limits_{\xi}^1 \frac{du}{u} g'(u) p_j(u) F(x_j)\\
h_2(\xi) &= \sum_{j,k} \int\limits_{\xi}^1 \frac{du}{u} g'(u) p_j(u) \int\limits_{x_j}^1 \frac{dz}{z} c(z) p_k(z) \cdot g(x_k)\\
h_2(\xi) &= \sum_{k} \left(\sum_j \int\limits_{\xi}^1 \frac{du}{u} g'(u) p_j(u) \left( \int\limits_{x_j}^1 \frac{dz}{z} c(z) p_k(z) \right)\right) \cdot g(x_k)
\end{align}

\section{References}
\begin{itemize}
  \item goharipour2020.pdf
  \item apfel: DIS.pdf
  \item  I. Schienbein et al., “A Review of Target Mass Corrections,” J. Phys. G 35, 053101 (2008).
  \item https://arxiv.org/pdf/0808.1231.pdf
  \item https://www.wiki.ed.ac.uk/display/nnpdfwiki/Theory+documentation
\end{itemize}

Furthermore Maria is the person who originally implemented TMC in NNPDF code, as a part of her PhD thesis.

\end{document}
